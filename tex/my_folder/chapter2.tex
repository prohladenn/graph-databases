\chapter{Выбор технологий для реализации графа знаний} \label{ch2}


\section{Подходы к хранению графовых данных}

Самой известной на сегодняшний день моделью данных, вероятно, является реляционная модель основанная Эдгаром Коддом в 1970 году. Данные
организованы в отношения (таблицы), где каждое отношение представляет собой неупорядоченную коллекцию кортежей (строк). Целью реляционной
модели было скрыть эту реализацию за чистым интерфейсом.

Поскольку данные хранятся в реляционных таблицах, между объектами в коде приложения и моделью базы данных требуется неудобный слой
трансляции: объектно-реляционное отображение (ORM). Это несоответствие между моделями называется импедансным несовпадением.

В середине 2000-х годов из-за следующих потребностей появилась новая нереляционная модель:

\begin{itemize}
    \item большая масштабируемость, так как реляционные базы данных не предназначены для горизонтального масштабирования;
    \item более гибкие схемы. Схемы являются полезным механизмом для документирования и принудительного применения этой структуры, однако подход "схема на чтение" (структура данных неявна и интерпретируется только при чтении данных) является выгодным, если элементы коллекции по каким-то причинам не имеют одинаковой структуры; произвольные ключи и значения могут быть добавлены в документ;
    \item лучшая локализация, по сравнению с многотабличной схемой; вся соответствующая информация находится в одном месте, и одного запроса достаточно для её извлечения.
\end{itemize}

На данный момент существует несколько типов NoSQL баз данных:

\begin{itemize}
    \item ключ значение;
    \item семейство столбцов;
    \item документоориентированная база данных;
    \item графовая база данных.
\end{itemize}

Большинство баз данных организованы на основе одной модели данных, однако существуют системы предназначенные для поддержки множества
моделей данных на основе единого интегрированного бэкэнда, такие базы данных называются мультимодельными.

Для проведение сравнение эффективности хранения и работы с графами знаний будут использованы базы данных различного типа, среди которых
присутствует реляционная, графовая и мультимодальная база данных.

\subsection{PostgreSQL}

PostgreSQL - это объектно-реляционная система баз данных с открытым исходным кодом, использующая и расширяющая язык SQL в сочетании со
многими функциями, которые позволяют безопасно хранить и масштабировать самые сложные данные. PostgreSQL заслужил прочную репутацию за
свою проверенную архитектуру, надежность, целостность данных, надежный набор функций, расширяемость и преданность сообществу с открытым
исходным кодом, стоящему за этим программным обеспечением, которое постоянно поставляет высокопроизводительные и инновационные решения.

PostgreSQL поставляется с большим количеством функций, которые помогают разработчикам создавать приложения, администраторам - защищать
целостность данных и создавать отказоустойчивые среды, а также управлять данными, независимо от их размера и размера. Помимо того, что
PostgreSQL является бесплатным приложением с открытым исходным кодом, он обладает широкими возможностями расширения. PostgreSQL старается
соответствовать стандарту SQL в тех случаях, когда такое соответствие не противоречит традиционным возможностям или может привести к
некачественным архитектурным решениям.

\subsection{Neo4j}

Neo4j - это NoSQL база данных с открытым исходным кодом, нативная база данных графов, которая обеспечивает ACID - совместимый бэкэнд
транзакций. Первоначальная разработка началась в 2003 году, но с 2007 года она стала общедоступной. Исходный код, написанный на Java и Scala,
доступен для бесплатного скачивания на GitHub или в виде удобного для пользователя приложения.

Neo4j называется нативной базой данных графов, поскольку она эффективно реализует модель графов свойств вплоть до уровня хранения.
В отличие от обработки графов или библиотек в памяти, Neo4j также предоставляет полные характеристики базы данных, включая соответствие
транзакциям ACID, поддержку кластера и аварийное переключение во время выполнения, что делает его пригодным для использования графов для
данных в производственных сценариях.

Некоторые из перечисленных ниже особенностей делают Neo4j очень популярным среди разработчиков, архитекторов и DBA:

\begin{itemize}
    \item Cypher, язык декларативных запросов, похожий на SQL, но оптимизированный для работы с графами;
    \item постоянное время прохождения на больших графах как в глубину, так и в ширину, благодаря эффективному представлению узлов и связей;
    \item гибкая схема графов свойств, которая может адаптироваться с течением времени, что позволяет материализовать и добавлять новые отношения позже, чтобы сократить и ускорить данные в домене, когда бизнес нуждается в изменении;
    \item драйверы для популярных языков программирования, включая Java, JavaScript, .NET, Python и многие другие.
\end{itemize}

\subsection{ArangoDB}

ArangoDB - это нативная мультимодельная база данных с открытым исходным кодом, предназначенная для хранения данных в виде пар ключ-значение,
графов и JSON документов, доступ к которым осуществляется с помощью одного декларативного языка запросов - AQL. С помощью ArangoDB можно
создавать высокопроизводительные приложения и масштабировать их по горизонтали, используя все три модели данных в полном объеме.
Кроме этого, ArangoDB делает акцент на высокой согласованности данных, упрощенное масштабирование и низкую совокупную стоимость владения.


\section{Обработка и извлечение данных}

\subsection{Задачи для проведения сравнения извлечения данных}

К основным типам извлечения данных в графах знаний относятся извлечение данных на основе атрибутов и классических графовых алгоритмов,
поэтому для сравнения функциональности и производительности языка запросов будет произведено сравнение извлечение данных на основе:

\begin{itemize}
    \item атрибутов сущности;
    \item атрибутов отношения;
    \item алгоритма поиска кратчайшего пути;
    \item алгоритма поиска ближайших соседей.
\end{itemize}

\subsection{Встроенные инструменты извлечения данных}

Neo4j предлагает встроенную в базу данных библиотеку “Graph Data Science Library” (GDSL), предоставляющую эффективно реализованные
параллельные версии распространенных графовых алгоритмов, встраиваемых в виде процедур языка запросов Cypher. Библиотека содержит
реализации следующих типов алгоритмов:

\begin{itemize}
    \item поиск пути - эти алгоритмы помогают найти кратчайший путь или оценить доступность и качество маршрутов;
    \item центральность - эти алгоритмы определяют важность отдельных узлов в сети;
    \item обнаружение сообщества - эти алгоритмы оценивают, как группа группируется или разделяется, а также как она усиливается или разбивается на части;
    \item сходство - эти алгоритмы помогают вычислить сходство узлов;
    \item link prediction - эти алгоритмы определяют близость пар узлов;
    \item node embeddings - эти алгоритмы вычисляют векторное представление узлов в графе;
    \item классификация узлов - данный алгоритм использует машинное обучение для прогнозирования классификации узлов;
\end{itemize}

ArangoDB предоставляет возможности языка AQL, в котором присутствуют операторы для решения задач всевозможных обходов графов, поиска
кратчайшего пути, а также операторы для поиска различного типа соседей (ближайшие, ближайшие k, все соседи).

\subsection{Использование сторонних библиотек для извлечения данных}

Так как среди баз данных представленных для сравнения присутствует база данных PostgreSQL, которая не предоставляет встроенную возможность
работы с графами и поддержку графовых алгоритмов, существует необходимость создания дополнительного сервисного слоя, который будет этим
заниматься. В качестве языка для сервиса был выбран язык Java из-за наличия большого количества библиотек для работы с графами и драйверов
для подключениями к базам данных.

На сайте www.baeldung.com, который содержит статьи и руководства по экосистеме Java присутствует статья, посвященная работе с графами, в
которой особое внимание уделается следующим библиотекам:

\begin{itemize}
    \item JGraphT - одна из самых популярных библиотек на Java для графовой структуры данных. Она позволяет создавать всевозможные виды графов и предлагает множество алгоритмов работы с графовыми структурами данных;
    \item Google Guava - это универсальный набор библиотек которые предлагают ряд функций, включая структуру данных графа и его базовые алгоритмы.
    \item Apache Commons - это проект Apache, который включает в себя Commons Graph, предлагая инструментарий для создания и управления структурой данных графа. Также предоставляет общие графовые алгоритмы для работы со структурой данных.
    \item Sourceforge Java Universal Network/Graph (JUNG) - это Java-фреймворк, который предоставляет расширяемый язык для моделирования, анализа и визуализации любых данных, которые могут быть представлены в виде графа. JUNG поддерживает ряд алгоритмов, которые включают такие процедуры, как кластеризация, декомпозиция и оптимизация.
\end{itemize}

Большинство из представленных библиотек подходят для решения поставленной задачи, но в своей работе “JGraphT - A Java library for graph
data structures and algorithms” авторы Димитриос Михаил и Йорис Кинабле подробно рассматривают реализацию структур данных в библиотеке
JGraphT и проводят сравнение скорости работы графовых алгоритмов этой библиотеки с библиотекой Jung, а также библиотекми NetworkX, BGL,
igraph реализованных на python и c++. По результатам их исследования на задаче поиска кратчайшего пути, по результатам представленным
на рис.2.10 видно, что JGraphT справляется с большими графами существенно лучше, чем Jung, поэтому именно библиотека JGraphT и будет
использована для работы с графами на сервисном уровне.

\begin{figure}[ht!]
    \center
    \includegraphics [scale=0.75] {my_folder/myimg//3}
    \caption{Сравнение скорости работы алгоритма поиска кратчайшего пути для разных библиотек}
\end{figure}


%Глава посвящена более подробным примерам оформления текстово-графических объектов.
%
%В параграфе \ref{ch2:title-abbr} приведены примеры оформления многострочной формулы и одиночного рисунка. Параграф \ref{ch2:sec-abbr} раскрывает правила оформления перечислений и псевдокода. В параграфе \ref{ch2:sec-very-short-title} приведены примеры оформления сложносоставных рисунков, длинных таблиц, а также теоремоподобных окружений.
%
%
%\section{Название параграфа} \label{ch2:title-abbr} %название по-русски
%
%
%
%%%%%
%%%
%%%  \input{...} commands are used only to sychronize some parts of the text with the author guide. Authors are free to type the text directly in .tex-files
%%%  \input{...} комманды используются только, чтобы синхронизировать части текта с рекомендациями авторам. Авторы  вольны вносить текст непосредственно в файл главы
%%%
% \input{my_folder/tex/eq-Galois} % пример двух выравнивания двух формул в окружении align
%
%
%На \firef{fig:spbpu-new-bld-autumn-ch2} приведёна фотография Нового научно-исследовательского корпуса СПбПУ.
%
%	\begin{figure}[ht]
%	\center
%	\includegraphics [scale=0.27] {my_folder/images/spbpu_new_bld_autumn}
%	\caption{Новый научно-исследовательский корпус СПбПУ \cite{spbpu-gallery}}
%	\label{fig:spbpu-new-bld-autumn-ch2}
%	\end{figure}
%
%
%
%
%\section{Название параграфа} \label{ch2:sec-abbr} %название по-русски
%
%Название параграфа оформляется с помощью команды \verb|\section{...}|, название главы --- \verb|\chapter{...}|.
%
%
%\subsection{Название подпараграфа} \label{ch2:subsec-title-abbr} %название по-русски
%
%
%Название подпараграфа оформляется с помощью команды  \texttt{\textbackslash{}subsection\{...\}}.
%
%
%%\subsubsection{Название подподпараграфа} \label{ch2:subsubsec-title-abbr} %название по-русски
%
%Использование подподпараграфов в основной части крайне не рекомендуется. В случае использования, необходимо вынести данный номер в содержание.
%Название подпараграфа оформляется с помощью команды  \texttt{\textbackslash{}subsubsecti\-on\{...\}}.
%
%
%
%\input{my_folder/tex/enumeration} % правила использования перечислений
%
%
%Оформление псевдокода необходимо осуществлять с помощью пакета \verb|algorithm2e| в окружении \verb|algorithm|. Данное окружение интерпретируется в шаблоне как рисунок. Пример оформления псевдокода алгоритма приведён на \firef{alg:AlgoFDSCALING}.
%
%
%\input{my_folder/tex/pseudocode-agl-DTestsFDScaling} % пример оформления псевдокода алгоритма
%
%
%\section{Название параграфа} \label{ch2:sec-very-short-title} %название по-русски
%
%
%
%\input{my_folder/tex/eq-equation-multilined} % пример оформления одиночной формулы в несколько строк
%
%\input{my_folder/tex/fig-spbpu-sc-four-in-one} % пример подключения 4х иллюстраций в одном рисунке
%
%%\input{my_folder/tex/fig-spbpu-whitehall-three-in-one} % пример подключения 3х иллюстрации в одном рисунке
%%
%%\input{my_folder/tex/fig-spbpu-main-bld-two-in-one} % пример подключения 2х иллюстраций в одном рисунке
%
%\input{my_folder/tex/tab-more-than-one-page} % пример подключения таблицы на несколько страциц
%
%
%\begin{table} [htbp]% Пример оформления таблицы
%	\centering\small
%	\caption{Пример представления данных для сквозного примера по ВКР \cite{Peskov2004}}%
%	\label{tab:ToyCompare}
%		\begin{tabular}{|l|l|l|l|l|l|}
%			\hline
%			$G$&$m_1$&$m_2$&$m_3$&$m_4$&$K$\\
%			\hline
%			$g_1$&0&1&1&0&1\\ \hline
%			$g_2$&1&2&0&1&1\\ \hline
%			$g_3$&0&1&0&1&1\\ \hline
%			$g_4$&1&2&1&0&2\\ \hline
%			$g_5$&1&1&0&1&2\\ \hline
%			$g_6$&1&1&1&2&2\\ \hline
%		\end{tabular}
%%	\caption*{\raggedright\hspace*{2.5em} Составлено (или/и рассчитано) по \cite{Peskov2004}} %Если проведена авторская обработка или расчеты по какому-либо источнику
%	\normalsize% возвращаем шрифт к нормальному
%\end{table}
%
%
%
%%% please, before using, read the author guide carefully
%
%\input{my_folder/tex/tab-toy-context-minipage} % пример подключения minipage
%
%\input{my_folder/tex/fig-spbpu-new-bld-autumn-minipage} % пример подключения minipage
%
%
%
%
%\input{my_folder/tex/rules-theorem-like-expressions}
%
%По аналогии с нумерацией формул, рисунков и таблиц нумеруются и иные текстово-графические объекты, то есть включаем в нумерацию номер главы, например: теорема 3.1. для первой теоремы третьей главы монографии. Команды \LaTeX{} выставляют нумерацию и форматирование автоматически. Полный перечень команд для подготовки текстово-графических и иных объектов находится в подробных методических рекомендациях \cite{spbpu-bci-template-author-guide}.
%
%
%\input{my_folder/tex/rules-list-of-environments} % список некоторых окружений
%
%
%\input{my_folder/tex/theorem-example} %пример оформления теоремы
%
%
%\input{my_folder/tex/definition-example} %пример оформления определения
%
%
%Вместо теоремо-подобных окружений для вставки небольших текстово-графических объектов иногда используются команды. Типичным примером такого подхода является команда \verb|\footnote{text}|\footnote{Внимание! Команда вставляется непосредственно после слова, куда вставляется сноска (без пробела). Лишние пробелы также не указываются внутри команды перед и после фигурных скобок.}, где в аргументе \verb|text| указывают текст \textit{подстрочной ссылки (сноски)}.В них \textit{нельзя добавлять веб-ссылки или цитировать литературу}. Для этих целей используется список литературы. Нумерация сносок сквозная по ВКР без точки на конце выставляется в шаблоне автоматически, однако в каждом приложении к ВКР нумерация, зависящая от номера приложения, выставляется префикс <<П>>, например <<П1.1>> --- первая сноска первого приложения.
%
%
%
%
%%\FloatBarrier % заставить рисунки и другие подвижные (float) элементы остановиться
%
%
%\section{Выводы} \label{ch2:conclusion}
%
%Текст заключения ко второй главе. Пример ссылок \cite{Article,Book,Booklet,Conference,Inbook,Incollection,Manual,Mastersthesis,Misc,Phdthesis,Proceedings,Techreport,Unpublished,badiou:briefings}, а также ссылок с указанием страниц, на котором отображены те или иные текстово-графические объекты  \cite[с.~96]{Naidenova2017} или в виде мультицитаты на несколько источников \cites[с.~96]{Naidenova2017}[с.~46]{Ganter1999}. Часть библиографических записей носит иллюстративный характер и не имеет отношения к реальной литературе.
%
%Короткое имя каждого библиографического источника содержится в специальном файле \verb|my_biblio.bib|, расположенном в папке \verb|my_folder|. Там же находятся исходные данные, которые с помощью программы \texttt{Biber} и стилевого файла \texttt{Biblatex-GOST} \cite{ctan-biblatex-gost} приведены в списке использованных источников согласно ГОСТ 7.0.5-2008.
%Многообразные реальные примеры исходных библиографических данных можно посмотреть по ссылке \cite{ctan-biblatex-gost-examples}.
%
%Как правило, ВКР должна состоять из четырех глав. Оставшиеся главы можно создать по образцу первых двух и подключить с помощью команды \verb|\input| к исходному коду ВКР. Далее в приложении \ref{appendix-MikTeX-TexStudio} приведены краткие инструкции запуска исходного кода ВКР \cite{latex-miktex,latex-texstudio}.
%
%В приложении \ref{appendix-extra-examples} приведено подключение некоторых текстово-графических объектов. Они оформляются по приведенным ранее правилам. В качестве номера структурного элемента вместо номера главы используется <<П>> с номером главы. Текстово-графические объекты из приложений не учитываются в реферате.
%


%% Вспомогательные команды - Additional commands
%
%\newpage % принудительное начало с новой страницы, использовать только в конце раздела
%\clearpage % осуществляется пакетом <<placeins>> в пределах секций
%\newpage\leavevmode\thispagestyle{empty}\newpage % 100 % начало новой страницы