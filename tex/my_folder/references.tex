%%% Не мянять - Do not modify
%%
%%
\clearpage                                  % В том числе гарантирует, что список литературы в оглавлении будет с правильным номером страницы
%\hypersetup{ urlcolor=black }               % Ссылки делаем чёрными
%\providecommand*{\BibDash}{}                % В стилях ugost2008 отключаем использование тире как разделителя 
\urlstyle{rm}                               % ссылки URL обычным шрифтом
\ifdefmacro{\microtypesetup}{\microtypesetup{protrusion=false}}{} % не рекомендуется применять пакет микротипографики к автоматически генерируемому списку литературы
%\newcommand{\fullbibtitle}{Список литературы} % (ГОСТ Р 7.0.11-2011, 4)
%\insertbibliofull  
%\noindent
%\begin{group}
\chapter*{Список использованных источников}	
\label{references}
\addcontentsline{toc}{chapter}{Список использованных источников}	% в оглавление 
%\printbibliography[env=SSTfirst]                         % Подключаем Bib-базы
%\ifdefmacro{\microtypesetup}{\microtypesetup{protrusion=true}}{}
%\urlstyle{tt}                               % возвращаем установки шрифта ссылок URL
%\hypersetup{ urlcolor={urlcolor} }          % Восстанавливаем цвет ссылок



%\urlstyle{rm}                               % ссылки URL обычным шрифтом
%\ifdefmacro{\microtypesetup}{\microtypesetup{protrusion=false}}{} % не рекомендуется применять пакет микротипографики к автоматически генерируемому списку литературы
%\insertbibliofull                           % Подключаем Bib-базы
%\ifdefmacro{\microtypesetup}{\microtypesetup{protrusion=true}}{}
%\urlstyle{tt}                               % возвращаем установки шрифта ссылок URL

1. R. Davis, H. Shrobe, and P. Szolovits. What is a Knowledge Representation? URL: http://groups.csail.mit.edu/medg/ftp/psz/k-rep.html (дата обращения: 20.10.2020).

2. Ehrlinger, Lisa and Wolfram Wöß. Towards a Definition of Knowledge Graphs, 2016.

3. Архитектура графа знаний. URL: https://miro.medium.com/max/2160/1*32FoLptjaXn2kbi867qI1Q.jpeg (дата обращения: 20.10.2020).

4. Building The LinkedIn Knowledge Graph. URL: https://engineering.linkedin.com/blog/2016/10/building-the-linkedin-knowledge-graph (дата обращения: 20.10.2020).

5. SPARQL. URL: https://www.w3.org/TR/rdf-sparql-query (дата обращения: 20.10.2020).

6. GraphQL. URL: https://graphql.org (дата обращения: 20.10.2020).

7. Agile Database Techniques: Effective Strategies for the Agile Developer. URL: http://www.ambysoft.com/books/agileDatabaseTechniques.html (дата обращения: 20.10.2020).

8. NoSQL. URL: https://ru.wikipedia.org/wiki/NoSQL (дата обращения: 20.10.2020).

9. PostgreSQL. URL: https://www.postgresql.org/about (дата обращения: 20.10.2020).

10. Neo4j. URL: https://neo4j.com/developer/graph-database/#neo4j-overview (дата обращения: 20.10.2020).

11. ArangoDB. URL: https://www.arangodb.com/ (дата обращения: 20.10.2020).

12. Neo4j Graph Data Science Library. URL: https://neo4j.com/developer/graph-data-science/graph-algorithms/#neo4j-algorithms-library (дата обращения: 20.10.2020).

13. AQL. URL: https://www.arangodb.com/docs/stable/aql/ (дата обращения: 20.10.2020).

14. Graphs in Java. URL: https://www.baeldung.com/java-graphs (дата обращения: 20.10.2020).

15. JGraphT - A Java library for graph data structures and algorithms. URL: https://www.researchgate.net/publication/332494171\_JGraphT\_--\_A\_Java\_library\_for\_graph\_data\_structures\_and\_algorithms (дата обращения: 20.10.2020).
