\chapter*{Введение} % * не проставляет номер
\addcontentsline{toc}{chapter}{Введение} % вносим в содержание

Количество информации, доступной сегодня в Интернете, поражает воображение, и оно постоянно расширяется. Например, существует более двух
миллиардов веб-сайтов, связанных с всемирной паутиной, и поисковые системы (например, Google, Bing и т.д.) могут просматривать эти ссылки и
предоставлять полезную информацию с большой точностью и скоростью. В большинстве этих успешных поисковых систем самым важным знаменателем
является использование графов знаний. Не только поисковые системы, но и сайты социальных сетей (например, Facebook и т.д.), сайты электронной
коммерции (например, Amazon и т.д.) также используют графы знаний для хранения и извлечения полезной информации.

Целью данной работы является исследование и сравнение инструментов для реализации и поддержки графов знаний. Для достижения данной цели необходимо решить следующие задачи:

\begin{itemize}
    \item исследовать предметную область;
    \item определить основные отличия в реализациях графов знаний;
    \item разработать задачи и метрики для проведения сравнения инструментов для реализации и поддержки графов знаний;
    \item разработать систему для проведения сравнения инструментов для реализации и поддержки графов знаний;
    \item произвести исследование на основе разработанных метрик с помощью реализованной системы;
    \item обработать результаты исследования;
    \item подвести итоги сравнения инструментов для реализации и поддержки графов знаний.
\end{itemize}

%% Вспомогательные команды - Additional commands
%\newpage % принудительное начало с новой страницы, использовать только в конце раздела
%\clearpage % осуществляется пакетом <<placeins>> в пределах секций
%\newpage\leavevmode\thispagestyle{empty}\newpage % 100 % начало новой строки