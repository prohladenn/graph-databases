\chapter*{Заключение}
\addcontentsline{toc}{chapter}{Заключение}

В ходе работы была исследована предметная область, связанная с графами знаний. Определены основные отличия, влияющие на реализацию и для
каждой реализации был подобран и описан стек технологий. Для проведения сравнения были разработаны задачи, метрики, а также интеграционная
система для работы с различными реализациями. Проведено исследование, по результатам которого каждая реализация получила описание своих
преимуществ и недостатков, а также области применения и оценку сложности реализации:

\begin{itemize}
    \item PostgreSQL — реляционная база данных подходит для работы с небольшими объемами графовых данных. К преимуществам этого подхода относятся: низкая потребность дисковой памяти и скорость работы на вставку и простое извлечение данных. К недостаткам этого подхода относятся: отсутствие поддержки графовых алгоритмов, необходимость самостоятельной реализации архитектуры и отсутствие инструментов для извлечения выгоды из графовой природы данных;
    \item Neo4j — графовая база данных подходит для максимально продуктивной работы для работы с графовыми данными любой степени сложности. К преимуществам этого подхода относятся: наличие всех необходимых инструментов для работы и извлечения выгоды из графовой природы данных. К недостаткам этого подхода относятся: высокая потребность дисковой памяти для хранения данных и скорость выполнения операция на вставку и извлечение данных;
    \item ArangoDB — мультимодельная база данных подходит для работы с графовыми данными данными любой степени сложности. К преимуществам этого подхода относятся: низкая потребность дисковой памяти, скорость работы на вставку и извлечение данных любой сложности. К недостаткам этого подхода относятся: отсутствие инструментов для продуктивной работы с графовой природой данных и слабая ориентированность встроенного языка запросов под графовые задачи.
\end{itemize}

По полученным результатам можно заключить, что все поставленные задачи были успешно решены, цель достигнута.