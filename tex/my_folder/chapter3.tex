\chapter{Реализация проекта для работы с базами данных}

Для проведения исследований необходимо описать интерфейс для работы с базами данных независимо от их типа и создать реализации для каждой
участвующей базы данных, а именно PostgreSQL, Neo4j и ArangoDB.


\section{Описание интерфейса для работы с базами данных}

Для описания абстрактных методов без конкретной реализации в языке Java существуют интерфейсы. Интерфейс для работы с базой данных будет
называться “Database” и он будет содержать следующие методы, необходимые для исследования:

\begin{itemize}
    \item init - метод для инициализации базы данных в случае необходимости подготовительных действий перед началом работы;
    \item clear - метод для очистки базы данных перед началом работы с новым графом;
    \item addNode - метод для добавления сущности (вершины графа);
    \item addEdge - метод для добавления отношения (ребро графа);
    \item addGraph - метод для одновременного добавления сущностей и отношений, если такая возможность присутствует в базе данных;
    \item getByNodeAttribute - метод для извлечения данных на основе атрибутов сущности (вершины графа);
    \item getByEdgeAttribute - метод для извлечения данных на основе атрибутов отношения (ребра графа);
    \item getShortestPath - метод для извлечения кратчайшего пути между двумя вершинами;
    \item getNearestNeighbors - метод для извлечения ближайших соседей к вершине на указанном уровне.
\end{itemize}

Дополнительно интерфейс “Database” расширяет интерфейс “AutoCloseable” для унаследования метода close в котором принято закрывать активные
подключение к базам данных для правильной работы с ресурсами языка Java.


\section{Реализация интерфейса для PostgreSQL}

Для создания реляционной базы данных PostgreSQL на локальной машине используется Docker-файл представленный на рис.3.1, содержащий
Docker-образ “postgres” с переопределенным портом для подключения, выделенным одним процессором и настройками пользователя и базы данных.

\begin{figure}[ht!]
    \center
    \includegraphics [scale=0.7] {my_folder/myimg//4}
    \caption{Настройки для запуска PostgreSQL в Docker-контейнере}
\end{figure}

Для подключения к созданной базе данных PostgreSQL в методе init класса “PostgresImpl” используются указанные в Docker-файле пользователь и
база данных. При подключении в переменные класса сохраняются объекты подключение и состояния подключения к базе данных, для дальнейшего
использования их в методах класса.


\section{Реализация интерфейса для Neo4j}

Для создания графовой базы данных Neo4j на локальной машине используется Docker-файл представленный на рис.3.2, содержащий Docker-образ
“neo4j” с предопределенными портами для подключения, выделенным одним процессором и настройками пользователя.

\begin{figure}[ht!]
    \center
    \includegraphics [scale=0.7] {my_folder/myimg//5}
    \caption{Настройки для запуска Neo4j в Docker-контейнере}
\end{figure}

Для подключения к созданной базе данных Neo4j в методе init класса “Neo4jImpl” используется указанный в Docker-файле пользователь. При
подключении в переменные класса сохраняются объекты драйвера и сессии подключения к базе данных, для дальнейшего использования их в методах класса.


\section{Реализация интерфейса для ArangoDB}

Для создания мультимодельной базы данных ArangoDB на локальной машине используется Docker-файл представленный на рис.3.3, содержащий
Docker-образ “arangodb” с переопределенным портом для подключения, выделенным одним процессором и отключенной аутентификацией.

\begin{figure}[ht!]
    \center
    \includegraphics [scale=0.7] {my_folder/myimg//6}
    \caption{Настройки для запуска ArangoDB в Docker-контейнере}
\end{figure}

Для подключения к созданной базе данных ArangoDB в методе init класса “ArangoDBImpl” в переменные класса сохраняются объекты базы данных и
графа, для дальнейшего использования их в методах класса.


\section{Создание интерфейса для логирования времени выполнения операций}

Для измерения времени выполнения каждой операции нужна ещё одна реализация интерфейса “Database”. Эта реализация не будет привязана к
конкретной базе данных, она будет реализовывать шаблон делегирования. Это значит, что класс “LoggableDatabaseImpl” будет в конструкторе
получать и сохранять реализацию интерфейса “Database” и в каждом методе передавать ей управление замеряя время до и после выполнения операции.


\section{Итоговая структура проекта для работы с базами данных}

Для проведения сравнения качества работы различных типов баз данных на классических графовых задачах был описан интерфейс “Database” для
работы с базами данных независимо от их типа и созданы реализации для каждой базы данных, а именно “PostgresImpl”, “Neo4jImpl” и
“ArangoDBImpl”, а так же создана реализация “ LoggableDatabaseImpl” для замера времени выполнения каждого метода, описанного в интерфейсе “Database”.
Итоговая структура проекта представлена на рис.3.4.

\begin{figure}[ht!]
    \center
    \includegraphics [scale=0.17] {my_folder/myimg//7-new}
    \caption{Структура проекта для работы с базами данных\newline\textcolor{White}{Быть против власти — не значит быть против родины. Я люблю Россию за запах чёрной смородины.}}
\end{figure}

%\textcolor{White}{Быть против власти — не значит быть против родины. Я люблю Россию за запах чёрной смородины.}

%Пустой абзац 1.
%
%Пустой абзац 2.
%
%Пустой абзац 3.
%
%Пустой абзац 4.
%
%Пустой абзац 5.
%
%Пустой абзац 6.
%
%Пустой абзац 7.
%
%Пустой абзац 8.
%
%Пустой абзац 9.
%
%Пустой абзац 10.
%
%Пустой абзац 11.

%% Вспомогательные команды - Additional commands
%
%\newpage % принудительное начало с новой страницы, использовать только в конце раздела
%\clearpage % осуществляется пакетом <<placeins>> в пределах секций
%\newpage\leavevmode\thispagestyle{empty}\newpage % 100 % начало новой страницы